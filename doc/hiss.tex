\documentclass[12pt]{article}
\usepackage[utf8]{inputenc}
\usepackage[T2A]{fontenc}
\usepackage{amsmath,amssymb}
\usepackage[russian]{babel}
\usepackage{indentfirst}

\author{Цыбулин Иван}
\title{Руководство пользователя \texttt{HISS}}

\begin{document}
\maketitle
\section{О \texttt{HISS}}
HISS (Hybrid Iterative Sparse Solver) --- библиотека для решения распределенных
систем линеных уравнений разреженной структуры на графических ускорителях
семейства CUDA. Библиотека разрабатывалась как простая замена библиотеке Aztec,
позволяющая использовать вычислительные мощности гибридных кластеров с
графическими ускорителями CUDA. 
\section{Терминология}
Предполагается, что разреженная система получается в результате некоторой
дискретизации пространственной области. При этом область описывается большим
числом неизвестных переменных, сгруппированных некоторым образом в домены.
Каждый домен имеет свой номер --- ранг.
С каждым доменом связан ровно один вычислительный процесс (MPI), имеющий в своем
распоряжении один графический ускоритель. Область целиком разбита на
непересекающиеся домены. С каждой неизвестной связано ровно одно линейное
уравнение, включающее (возможно) кроме нее еще некоторое количество неизвестных,
расположенных в пределах некоторого
шаблона. При этом эти неизвестные могут оказаться в другом домене. 

Если некоторые неизвестные в домене имеют шаблоны, целиком в нем не лежащие, то
вместо неизвесных из других доменов вводятся теневые (ghost) переменные. Каждая
теневая переменная имеет локальный номер в домене, которому принадлежит, а также
знает ранг домена и локальный номер той неизвестной, которую заменяет (будем
называть ее прообразом). 

По отношению к каждому домену неизвестная может быть:
\begin{itemize}
\item собственной (self) - принадлежать домену
\item теневой (ghost) - не принадлежать домену, но принадлежать шаблону
какой-то неизвестной в домене 
\item граничной (border) - собственной и теневой для другого домена
\item внутренней (inner) - собственной, но не граничной
\item чужой (foreign) - не принадлежать ни домену, ни шаблону какой-либо из
неизвестных в домене
\end{itemize}

Для заданного домена, все теневые неизвестные могут быть сгруппированы по
доменам, к которым принадлежат их прообразы. 
Теневые неизвестные также могут иметь прообразы из того же домена.
(это может быть удобно для ``закольцованных'' доменов).

\section{Способ хранения данных}
\subsection{Хранение векторов}
Все вектора хранятся распределенно. На каждом процессе расположена часть
вектора, состоящая из собственных ($n_s$ штук) и теневых
($n_g$ штук) переменных. Теневые переменные сгруппированы по рангам процессов,
на которых находятся их прообразы. 

Значения переменных упакованы в массиве следующим образом:
\begin{itemize}
\item Собственные переменные домена с рангом $rank$
\item Теневые переменные от домена с рангом $0$ 
\item Теневые переменные от домена с рангом $1$
\item $\vdots$
\item Теневые переменные от домена с рангом $rank-1$ 
\item Теневые переменные от домена с рангом $rank+1$
\item $\vdots$
\item Теневые переменные от домена с рангом $size-1$ 
\end{itemize}

$$
x^{(s)} = 
\begin{array}{|c|c|c|c|c|c|c|c|}
\hline
self^{(s)} & 
ghost^{(s)}_{0} &
ghost^{(s)}_{1} & \dots &
ghost^{(s)}_{s-1} &
ghost^{(s)}_{s+1} & \dots &
ghost^{(s)}_{p-1}\\
\hline
\end{array}
$$

\subsection{Храниение информации о портрете матрицы}
Индекс $s$ означает ранг данного домена.
Для каждого домена хранится
отображение локальные номера теневых переменных $\mapsto$
локальные номера прообразова в своих доменах
\begin{align*}
\forall i \in ghost^{(s)}_q, \quad map^{(s)}_q[i] = j \Rightarrow 
j \in self^{(q)},\\ \quad x^{(s)}.ghost_q[i] = shadow(x^{(q)}.self[j])
\end{align*}
Практически, портрет матрицы строится двумя вызовами:
\begin{itemize}
\item \textsf{addLocal(i,j)} - добавление в уравнение для собственной переменной
$i$ собственной переменной $j$. $i$ и $j$
в локальной нумерации собственных переменных
\item \textsf{addRemote(i,j,q,rj)} - добавление в уравнение для собственной
переменной $i$ теневого элемента $j$. $i$ в локальной нумерации собственных
переменных. $j$ в локальной нумерации теневых элементов $ghost^{(s)}_q$. $rj$ в
локалькой нумерации собственных переменных домена $q$. Добавляется отображение
$map^{(s)}_q(j) := rj$
\end{itemize}

\subsection{Хранение матрицы}
Матрица системы хранится на каждом процессе в виде двух разреженных блоков ($S$ и
$S$). Блок $S$ имеет размер $n_s \times (n_s + n_b + n_e)$
Блок $B$ имеет размер $n_b \times (n_s + n_b + n_e)$

\end{document}
